\documentclass{article}

\usepackage{fancyvrb}
\usepackage{appendix}
\usepackage[french]{babel}
\usepackage[T1]{fontenc}
\usepackage[style=french]{csquotes}
\usepackage[margin=3cm]{geometry}
\usepackage{graphicx}
\usepackage{listings}
\usepackage{xifthen}
\usepackage{hyperref}
\usepackage{caption}

% pour du code verbatim
\lstnewenvironment{java}{\lstset{language=Java,basicstyle=\small}}{}
\lstnewenvironment{shellcmds}{\lstset{language=Java,basicstyle=\small}}{}

% abréviations latine
\newcommand{\etal}{\textit{et al.}\@\xspace}
\newcommand{\ie}{\textit{i.e.}\@\xspace}
\newcommand{\eg}{\textit{e.g.}\@\xspace}

% pour que make/latexmk genère les fichiers manquantes
\newcommand{\makeincudegraphics}[2][]{%
    \IfFileExists{#2}{\includegraphics[#1]{#2}}{\typeout{No file #2.}}}

% pour avoir \verb dans les notes de bas de page
\VerbatimFootnotes

\title{Rapport SAE}
\author{BA GUBAIR Emad \and BARTOS Tom \and MARÇAIS André \and ROBLIN Yann}

\begin{document}
\maketitle{}

Le Maitre du Jeu (MDJ) gère les drones virtuelles et peut modifier leurs environnement.
Le \emph{Système de Pilotage} (figure \ref{fig:uc})
est implanter dans les figures
\ref{fig:fwclient} et \ref{fig:fwserveur}.
%
\begin{figure}
    \makeincudegraphics[width=\textwidth]{simulateur_uc.png}
    \caption{Diagramme de cas d'usage}
    \label{fig:uc}
\end{figure}
%
\begin{figure}
    \makeincudegraphics[width=\textwidth]{ardupilot_client_components.png}
    \caption{Diagramme de composantes (micrologiciel coté client)}
    \label{fig:fwclient}
\end{figure}
%
\begin{figure}
    \makeincudegraphics[width=\textwidth]{ardupilot_server_components.png}
    \caption{Diagramme de composantes (micrologiciel coté serveur)}
    \label{fig:fwserveur}
\end{figure}
%
\begin{figure}
    \makeincudegraphics[width=\textwidth]{classes.png}
    \caption{Diagramme de classes}
    \label{fig:classes}
\end{figure}

\end{document}
