\documentclass{scrartcl}

\usepackage{fancyvrb}
\usepackage{appendix}
\usepackage[french]{babel}
\usepackage[T1]{fontenc}
\usepackage[style=french]{csquotes}
\usepackage[margin=2cm]{geometry}
\usepackage{graphicx}
\usepackage{listings}
\usepackage{xifthen}
\usepackage{hyperref}
\usepackage{caption}

% pour du code verbatim
\lstnewenvironment{java}{\lstset{language=Java,basicstyle=\small}}{}
\lstnewenvironment{shellcmds}{\lstset{language=Java,basicstyle=\small}}{}

% abréviations latine
\newcommand{\etal}{\textit{et al.}\@\xspace}
\newcommand{\ie}{\textit{i.e.}\@\xspace}
\newcommand{\eg}{\textit{e.g.}\@\xspace}

% pour que make/latexmk genère les fichiers manquantes
\newcommand{\makeincudegraphics}[2][]{%
    \IfFileExists{#2}{\includegraphics[#1]{#2}}{\typeout{No file #2.}}}

% pour avoir \verb dans les notes de bas de page
\VerbatimFootnotes

\title{Rapport SAE}
\author{BA GUBAIR Emad \and BARTIER Tom \and MARÇAIS André \and ROBLIN Yann}

\begin{document}

\maketitle{}

\begin{abstract}
    Ce document est un rapport qui présente le projet SAE du master 1 informatique et mathématique 2024 - 2025
    de l'Université de Toulon. Il est encadré Julien Seinturier et Emmanuel Bruno, professeurs à l'Université de Toukon.
    Le projet consiste à développer un simulateur de drone sous-marin en Java, pour cela les étudiants
    ont utilisé les connaissances acquises tout au long de l'année, notemment en programmation, en gestion de projet et
    en modélisation.
\end{abstract}

\section{Problématique}
Les fonds marins sont un milieu que nous connaissons peu et qui sont difficiles d'accès.
Afin de mieux les étudier, nous pouvons utiliser des drones sous-marins pilotés à distance
afin de mener différentes missions comme la prise de photos ou la collecte de données.
Cependant, les drones sous-marins sont coûteux et demandent un minimum d'entrainement au pilotage.
Pour des missions très délicates on pourrait imaginer qu'une IA pourrait réaliser ces missions, mais cela
nécessite de pouvoir entrainer l'IA, notemment dans un environnement virtuel. L'idée de faire un simualteur
est donc vraiment pertinente car il permettrait d'entrainer à la fois des pilotes humains et des IA.

\section {Description générale du projet et fonctionnalités}

Ardupilot est une interface de pilotage de drone open source utilisée par de nombreux passionnés et professionnels
pour piloter tout type de drone (aérien, sous-marin ou terrestre). L'objectif de ce projet est de développer un simulateur
de drone sous-marin que l'on pourrait connecter à Ardupilot un peu comme si c'était un vrai drone afin que des pilotes ayant de l'expérience
avec Ardupilot puissent passer du drone virtuel au drone réel sans avoir à changer leurs habitudes. Afin d'aller un petit peu plus loin et de permettre
un certaine flexibilité, le simulateur a été conçu pour pouvoir être facilement adapté à n'importe quelle interface de pilotage.

Le simulateur a été écrit en Java, il contient plusieurs composantes (voir figure \ref{fig:fwclient} et \ref{fig:fwserveur}).
Tout d'abord, on peut identifier 4 acteurs principaux :
\begin{itemize}
    \item Le pilote, qui pilote le drone virtuel.
    \item Le maître du jeu (MDJ), qui gère les drones virtuels et peut modifier leurs environnement.
    \item L'observateur, qui peut observer l'environnement de la simulation
    \item L'administrateur, qui gère les utilisateurs
\end{itemize}

Pour en savoir plus sur les acteurs, vous pouvez consulter le diagramme de cas d'usage (voir figure \ref{fig:uc}).
La simulation tourne autours d'une application serveur REST appelé ici "Manager".
Le Manager gère principalement l'authentification des utilisateurs, la création des drones virtuels, le lancement du serveur de simulation
et l'ajout/suppression en temps réel d'évènements dans l'environnement, et dans le futur l'ajout en temps réel de nouveaux drones virtuels.
Les utilisateurs disposent d'un logiciel avec une interface graphique appelé ici "launcher" qui leur permet de se connecter et d'interagir avec la simualtion
en fonction de leur rôle. Concernant les pilotes, le launcher est aussi responsable du lancement des différents composants de l'interface de pilotage.


\begin{figure}
    \makeincudegraphics[width=1.1\textwidth]{ardupilot_server_components.png}
    \caption{Diagramme de composantes (rendu côté serveur)}
    \label{fig:fwclient}
\end{figure}

\begin{figure}
    \makeincudegraphics[width=1.1\textwidth]{ardupilot_client_components.png}
    \caption{Diagramme de composantes (rendu côté client)}
    \label{fig:fwserveur}
\end{figure}

\begin{figure}
    \makeincudegraphics[width=\textwidth]{simulateur_uc.png}
    \caption{Diagramme de cas d'usage}
    \label{fig:uc}
\end{figure}

\section {Brouillon}
Le Maitre du Jeu (MDJ) gère les drones virtuelles et peut modifier leurs environnement.
Le \emph{Système de Pilotage} (figure \ref{fig:uc})
est implanter dans les figures
\ref{fig:fwclient} et \ref{fig:fwserveur}.




Le serveur d'authentification, le simulateur, et le renderer (quand il est coté serveur)
partagent les clés de chiffrement et l'adresse des flux vidéo comme nécessaire.
%
\begin{figure}
    \makeincudegraphics[width=\textwidth]{conn_pilote_seq.png}
    \caption{Connexion d'un pilote avec rendu à distance.}
\end{figure}

L'interface \verb|Controler| correspond au micro contrôleur du drone virtuelle.
La classe \verb|ArduSubControler| gère les messages MAVLink envoyé par une instance du ArduSub SITL.
Le drone virtuelle appelle au contrôleur pour appliquer les forces/torseurs dynamiques
(\enquote{screw} en anglais)
sur le drone.
Chaque drone a un \verb|Renderer| pour l'afficher dans la scène.
Le rendu est ensuite envoyé sur le réseau.
%
\begin{figure}
    \centering
    \makeincudegraphics[width=0.8\textwidth]{classes.png}
    \caption{Diagramme de classes}
    \label{fig:classes}
\end{figure}

\end{document}
